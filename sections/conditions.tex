\startcomponent conditions
\product thesis
\environment thesislayout


\section[sec:conditions]{Task conditions}

For expression equivalence, we needed contexts to determine the equivalence of two lambda functions, whose results can only be observed after evaluation.
For tasks however, we do not need contexts to view their results.
A task's value can be determined at any point during execution by using the $\Value$ observation,
whereupon it either has a value, or it is undefined.
Note that, we will restrict ourselves to tasks which result in a basic value,
as these are results which are directly comparable.

On the other hand, tasks do allow user interaction, and depending on what inputs are sent, the resulting task may be different.
So while lambda functions can produce different results depending on their arguments, so can tasks produce different results depending on what inputs are send to them.
So, in a sense, for tasks the contexts are user input.

To satisfy the first property at the beginning of \doifthesiselse{this chapter}{\see[sec:contextequiv]},
that programs resulting in values which are observably different must not be equivalent,
we will use the observations on tasks, as introduced in \see[sec:observations].
Before we observe a task however, we should first fully fixate the tasks whose equivalence we want to determine.
We use the fixing semantics ($\fixate$) to ensure that tasks are fully fixated.
Fixation keeps track of a state $\sigma$.
For semantic equivalence, we do not want that fixation results in two different states.
So, given two tasks $t_1, t_2 : \Task\ \beta$, we want that for all states $\sigma$, fixation of $t_1$ and $t_2$ end in the same state $\sigma'$:
\startformula
  t_1,\sigma,\nothing \fixate t_1',\sigma' \iff t_2,\sigma,\nothing \fixate t_2',\sigma'
\stopformula

After fixation, we need to decide what observations $t_1'$ and $t_2'$ must have in common for them to be considered equivalent.
Let us recall what observations can be made on tasks.
The value function $\Value$ returns the value $v$ of a task, or $\bot$ if it is undefined;
there is the failing function $\Failing$ which returns whether a task is failing;
and there is the inputs function $\Inputs$ which returns the set of all possible input events that a task accepts,
that is, their \emph{input space}.
The value and failing functions are used in the normalisation rules of \TOPHAT, and different observations for these functions can result in different derivation rules being triggered.
Therefore, we can say that tasks for which the value or failing function return a different result must not be semantically equivalent.

Similarly, tasks whose inputs function $\Inputs$ return a different set of input events can also not be semantically equivalent, because that would mean that the types of interaction that can be done with the tasks are different.
Recall that input events should be named by the same key as the editor it is meant for.
So if we require that $\Inputs(t_1') = \Inputs(t_2')$,
then this also implies that $t_1'$ and $t_2'$ must have the same keys for all their editors.
If this is not the case, we can never have that $\Inputs(t_1') = \Inputs(t_2')$, and the tasks cannot be semantically equivalent.

Given these impressions, we say that at least the following property must hold for $t$ to be equivalent to $t'$:
\startformula
   \Failing(t) = \Failing(t') \text{\ and\ } \Value(t,\sigma) = \Value(t',\sigma) \text{\ and\ } \Inputs(t) = \Inputs(t')
\stopformula
For any of these task observations, we can distinguish two cases: either a task fails or does not fail, it either has a value or its value is undefined, and it either accepts input or it accepts no more input.
Based on these case distinctions, we say that a task is in either one of five conditions after fixation.
% \startdefinition[def:conditions]{Condition}
%   We use the observations $\Failing$, $\Value$, and $\Inputs$
%   to define five \emph{conditions} tasks can be in.
%   \startitemize[n]
%     \item \cnd{Failing}:
%       its failing observation is true
%       \NL $\Failing(n) = \True$;
%     \item \cnd{Finished (steady)}:
%       it has a value, but does not accept user input;
%       \NL $\Value(n,\sigma) = v$ and $\Inputs(n) \eq \emptyset$;
%     \item \cnd{Finished (recurring)}:
%       it has a value, and accepts user input
%       \NL $\Value(n,\sigma) = v$ and $\Inputs(n) \neq \emptyset$;
%     \item \cnd{Stuck}:
%       it has no value, and does not accept user input.
%       \NL $\Value(n,\sigma) = \bot$ and $\Inputs(n) \eq \emptyset$;
%     \item \cnd{Running}:
%       it has no value, and accepts user input;
%       \NL $\Value(n,\sigma) = \bot$ and $\Inputs(n) \neq \emptyset$;
%   \stopitemize
%   These conditions, are summarized in \see[tab:conditions],
%   including some examples tasks falling into that condition.
% \stopdefinition
These task conditions, and some examples, are shown in \see[tab:task-conditions].
In the next subsections, we will discuss each condition in more detail and give some examples.

\placetable[][tab:task-conditions]
  {Conditions for fixated tasks}% tasks can be in after normalisation}
  {\doifthesiselse{
    \starttabulate[|cM|cM|cM|l|lM|]
      \FL
      \NC \Failing   \NC \Value     \NC \Inputs    \NC \alert{Condition} \NC \alert{Examples}                                                                                                           \NR
      \ML
      \NC \checkmark \NC -          \NC -          \NC Failing           \NC \Fail, \,\, \Fail \Pair \Fail, \,\, \Fail \Choose \Fail, \,\, (\lambda x.x) \Trans \Fail, \,\, \Fail \Step \lambda x.\Lift x \NR
      \NC -          \NC \checkmark \NC -          \NC Steady            \NC \Lift 2, \,\, \Lift 2 \Pair \Lift 3, \,\, \Lift \tuple{2, 3}, \,\, (\lambda x.x + 1) \Trans \Lift 2                          \NR
      \NC -          \NC \checkmark \NC \checkmark \NC Unsteady          \NC \Update 2, \,\, \Update 2 \Pair \Update 3, \,\, \Update \tuple{2, 3}                                                       \NR
      \NC -          \NC -          \NC \checkmark \NC Running           \NC \Enter \Int \Pair \Fail, \,\, \Update 2 \Step \lambda x.\Fail                                                              \NR
      \NC -          \NC -          \NC -          \NC Stuck             \NC \Lift 2 \Step \lambda x.\Fail, \,\, \Lift 2 \Pair \Fail, \,\, \Fail \Pair \Lift 2                                          \NR
      % \NC - \NC - \NC \checkmark \NC Running (branching) \NC \Enter \Int \Step \lambda x.\Ite{x \leq 2}{\Lift x}{\Fail} \NR
      \LL
    \stoptabulate
    \noindentation
    (Checkmarks (\checkmark) for $\Failing$, $\Value$ and $\Inputs$ for a task $t$ and a state $\sigma$ indicate that
      $\Failing(t) = \True$,
      $\Value(t, \sigma) = v$ for $v \neq \bot$, and
      $\Inputs(t) \neq \nothing$ respectively.)
  }{\begin{tabular}{CCClL}
  \toprule
  \Failing   & \Value     & \Inputs    & \alert{Condition} & \alert{Examples}                                                                                                             \\
  \midrule
  \checkmark & -          & -          & Failing           & \Fail, \,\, \Fail \Pair \Fail, \,\, \Fail \Choose \Fail, \,\, (\lambda x.x) \Trans \Fail, \,\, \Fail \Step \lambda x.\View x \\
  -          & \checkmark & -          & Finished steady   & \View 2, \,\, \View 2 \Pair \View 3, \,\, \View \tuple{2, 3}, \,\, (\lambda x.x + 1) \Trans \View 2                          \\
  -          & \checkmark & \checkmark & Finished unsteady & \Update 2, \,\, \Update 2 \Pair \Update 3, \,\, \Update \tuple{2, 3}                                                         \\
  -          & -          & \checkmark & Running           & \Enter \Int \Pair \Fail, \,\, \Update 2 \Step \lambda x.\Fail                                                                \\
  -          & -          & -          & Stuck             & \View 2 \Step \lambda x.\Fail, \,\, \View 2 \Pair \Fail, \,\, \Fail \Pair \View 2                                            \\
  \bottomrule
\end{tabular}
\\
{\small
  Checkmarks for $\Failing$, $\Value$ and $\Inputs$ for a task $t$ and a state $\sigma$ indicate that\\
  $\Failing(t) = \True$,
  $\Value(t, \sigma) = v$ for $v \neq \bot$, and
  $\Inputs(t) \neq \nothing$ respectively.
}
}}


\subsection{Failing tasks}

A \emph{failing} task $t$ is a task for which the failing function $\Failing(t)$ yields true%
\doifthesis{, as stated before in \see[def:failing-task]}.

In the original \TOPHAT\ paper \cite[conf/ppdp/SteenvoordenNK19],
Theorem\~6.5 states that a task fails if and only if it accepts no more user input.
However, with the introduction of the lift combinator ($\Lift$) in \doifthesiselse{this thesis}{Steenvoorden \cite[Steenvoorden22]},
this is no longer the case.
For example, $\Lift e$ does not fail, and neither does it accept user input.
The same holds for read-only editors ($\View,\Watch$).
What we still can say however, is that if a task is failing, we know that it has no value and accepts no more user input.
We have proven this property in \see[prp:failing-no-interaction]\doifpaper{ in the appendix}.

% \startconjecture[con:failing]
%   For all failing tasks $t$ and states $\sigma$, we have that $\Value(t, \sigma) = \bot$ and $\Inputs(t) = \nothing$.
%   % Furthermore, $t$ cannot be further fixated.
%   % That is, for all states $\sigma$, we have that $t,\sigma,\nothing \fixate t,\sigma$.
%   % A failing task $t$ has no value and accepts no user input:
%   % $$\forall t \in \mathrm{Tasks}: \Failing(t) \rightarrow (\forall \sigma \in \mathrm{States}: \Value(t,\sigma) = \bot) \wedge \Inputs(t) = \nothing$$
% \stopconjecture

% \startconjecture[con:n-failing]
%   A failing task $t$ is fully fixated:
%   $$\Failing(t) \Rightarrow \forall \sigma \in \mathrm{States}: t,\sigma,\nothing \fixate t,\sigma$$
% \stopconjecture

\startproposition[prp:failing-stays-failing]{Failing tasks stay failing}
  If $t$ is a failing task, then it stays failing.
  Or, more formally:
  for all fixated tasks $t : \Task\ \tau$ and states $\sigma$,
  % that is $\Gamma,\Sigma \infers n:\Task\tau$ and $\Gamma,\Sigma \infers \sigma$,
  we have that
  \NL if $\Failing(t)$,
  \NL there is no input $\iota$ such that $t,\sigma \interact{\iota} t',\sigma''$.
\stopproposition

\startproof
  Once a task fails, it will always remain failing,
  because by \see[cor:failing-tasks-no-handling] \doifpaper{from the appendix} no more user interaction is possible, and by assumption, the task is already fixated.
\stopproof

Failing can thus be regarded as one type of termination, and we will consider all tasks that fail to be equivalent.
Hence, we will say that the tasks $\Fail$, $\Fail \Pair \Fail$, $\Fail \Choose \Fail$, $(\lambda x.\ x) \Trans \Fail$, $\Fail \Step \lambda x.\ \View x$, and all other failing tasks, are semantically equivalent to each other.


\subsection{Finished tasks}

A \emph{finished} task $t$ is a task which yields a value $\Value(t,\sigma) = v$ for $v \neq \bot$.
This value can either be \emph{steady} when no more user input is possible, or \emph{unsteady} when the task still accepts user input.
An example of a finished task with a steady value is the task $\View^k 42$.
Because this task accepts no more user input, its value will always remain equal to $42$.
An example of a finished task with an unsteady value is the task $\Update^k 42$.
This task still accepts user input, and thus its value can keep on changing.
Even though both tasks yield the same value, they should not be equivalent, since one's value can be changed and the other one's value cannot.
\doifthesis{We extend \see[def:finished] in the following way.}
\startdefinition[def:finished-task]{Finished, steady, and unsteady task}
  We call a fixated task $t : \Task\ \tau$ in state $\sigma$ \emph{finished} if and only if $\Value(t,\sigma) = v$, for $v \neq \bot$.
  Furthermore, we call a finished task \emph{steady} if and only if $\Inputs(t) = \nothing$, and \emph{unsteady} in the other case.
\stopdefinition

A steady task can thus never be semantically equivalent to an unsteady task, even if their values are (initially) the same.
But just looking at the resulting values, and whether the resulting value is steady or not, is not enough to determine semantic equivalence of finished tasks.
Consider for example the following tasks:
\startformulas[spread]
  \NC t_1 \NC:=\NC \View \tuple{2, 3}      \NC   t_3 \NC:=\NC \Update^{k_1} \tuple{2, 3}            \NR
  \NC t_2 \NC:=\NC \View 2 \Pair \View 3   \NC   t_4 \NC:=\NC \Update^{k_1} 2 \Pair \Update^{k_2} 3 \NR
\stopformulas

These are all finished tasks with value $\tuple{2, 3}$.
Tasks $t_1$ and $t_2$ are both steady, and thus $t_1 \taskequiv t_2$.
We have already concluded that steady tasks cannot be equivalent to unsteady tasks, so $t_1 \ntaskequiv t_3$, $t_1 \ntaskequiv t_4$, $t_2 \ntaskequiv t_3$, and $t_2 \ntaskequiv t_4$.
But we will also say that $t_3 \ntaskequiv t_4$, because we have that:
\startformulas[align]
  \NC \Inputs(t_3) \NC=\NC \set{k_1!b_1 \mid b_1: \Int \times \Int}                         \NR
  \NC \Inputs(t_4) \NC=\NC \set{k_1!b_1 \mid b_1 : \Int} \cup \set{k_2!b_2 \mid b_2 : \Int} \NR
\stopformulas
\noindentation
Meaning that the types of interaction that can be done with $t_3$ differ from the types of interaction that can be done with $t_4$.
In the case of $t_3$, the user can only alter the pair in one go, whereas for $t_4$, the user can partially update it.
So for finished tasks, we also need to require that $\Inputs(t) = \Inputs(t')$ if we want to conclude that $t \taskequiv t'$.

And yet this is still not enough...
Consider the following two tasks:
\startformulas[align]
  \NC t_5 \NC:=\NC (\lambda x.\ \Ite{x < 0}{-x}{x}) \Trans \Update^k 42 \NR
  \NC t_6 \NC:=\NC \Update^k 42                                         \NR
\stopformulas
\noindentation
Both tasks have the value $42$ and accept the same input.
However for negative values, the resulting values diverge, because the transform function in task $t_5$ ensures that its output is always positive.
So if a task still accepts user input, not only do we need to look at a task's current value, but we also need to consider all values that a task can have after user interaction.

We will regard finished tasks as another type of condition a task can be in after fixation.
Of course, if the value is unsteady, the task is not terminated in the true sense of the word, since there is still user interaction possible.
In fact, the task will never terminate: a finished task will always remain a finished task with the same input space, only its value may change.

\startproposition[prp:finished-stays-finished]{Finished tasks stay finished}
  If $t$ is a finished task, then for all inputs $\iota \in \Inputs(t)$: if $t,\sigma \interact{\iota} t',\sigma'$, then $t'$ is again a finished task.
  Moreover, we also have that $\Inputs(t') = \Inputs(t)$.
\stopproposition

\startproof
  Because $t$ is finished, it is fixated and has a value by \see[def:finished-task].
  Therefore, by \see[prp:valued-means-static]\doifpaper{ from the appendix}, we know that $t$ is a \emph{static} task.
  That is, it is a task which does not change the combinators it utilises.
  Using the properties of static tasks, as proved in \doifthesiselse{\see[sec:interplay]}{the appendix},
  we know $t$ stays static by \see[cor:static-stays-static],
  and its inputs will stay the same over interactions by \see[cor:static-keeps-input].
\stopproof

% NOT TRUE!
% We also claim that for finished tasks, it is enough to consider all values after a single input event, and that we do not need to
% For finished unsteady tasks, we claim that it is enough to
% \towrite{...}
% \towrite{Lemma: If $\Value(t,\sigma) = v$ for $v \neq \bot$, then for all $i \in \Inputs(t)$: if $t,\sigma \interact{i} t',\sigma'$ then $\Value(t',\sigma') = v'$ with $v' \neq \bot$ and $\Inputs(t) = \Inputs(t')$. I.e.: a finished task will always remain finished with the same input space, only its value may change.}
% \startconjecture[con:finished2]
%   If $t$ is a finished task, then for all valid input sequences $I$ with $I = i_0, ..., i_n$:
%   %
%   $$t,\sigma \interact{I} t',\sigma' \iff t,\sigma \interact{i_n} t',\sigma'$$
% \stopconjecture


\subsection{Stuck tasks}

A \emph{stuck} task $t$ is a task which does not fail, does not have a value, and does not accept user input.
% Such tasks are essentially broken.
Examples of stuck tasks are $\View 2 \Step \lambda x.\ \Fail$, $\View 2 \Pair \Fail$, and $\Fail \Pair \View 2$.
The first example is stuck because the right-hand side always fails, and thus the step can never be taken.
For pairing, we have that both sides must fail before $\Pair$ fails, and both sides must have a value before $\Pair$ has a value.
So, if one side fails and the other side has a value, then neither observation is true, and the task is stuck.

\startdefinition{Stuck task}
  We call a fixated task $t : \Task\ \tau$ in state $\sigma$ \emph{stuck} if and only if $\neg \Failing(T)$, $\Value(t,\sigma) = \bot$, and $\Inputs(t) = \nothing$.
\stopdefinition

A stuck task will always remain stuck, because no more user interaction is possible, and by assumption it is already fully fixated.
\startproposition[cor:stuck-stays-stuck]{Stuck tasks stay stuck}
  If $t$ is a stuck task, then it stays stuck.
  Or, more formally:
  for all fixated tasks $t : \Task\ \tau$ and states $\sigma$,
  % that is $\Gamma,\Sigma \infers n:\Task\tau$ and $\Gamma,\Sigma \infers \sigma$,
  we have that
  \NL if $\Inputs(t) = \nothing$,
  \NL there is no input $\iota$ such that $t,\sigma \interact{\iota} t',\sigma'$.
\stopproposition

Similar to failing, we will consider stuck tasks as another type of termination, and we will say that all stuck tasks are semantically equivalent to each other.


\subsection{Running tasks}

A \emph{running} task $t$ is a task which does not fail, does not have a value, but still accepts user input.
Because there is still user interaction possible, it may be the case that with the right input, it transitions to one of the previously described task conditions.
The simplest example of a running task is the empty editor $\Enter^k \beta$, which becomes a finished (unsteady) task once it receives a valid input event.
There also exist running tasks that only transition to another task condition for some inputs, or for no inputs at all.
For example, the tasks $\Enter^k \Int \Step \lambda x.\ \Fail$, $\Enter^k \Pair \Fail$, and $\Fail \Pair \Update^k 2$ are all running tasks which will forever remain running;
and the task $\Enter^k \Int \Step \lambda x.\ \Ite{x \leq 2}{\View x}{\Fail}$ will only transition to another task condition for some inputs, but not for others.

\startdefinition{Running task}
  We call a fixated task $t : \Task\ \tau$ in state $\sigma$ \emph{running}
  if and only if $\neg \Failing(T)$, $\Value(t,\sigma) = \bot$, and $\Inputs(t) \neq \nothing$.
\stopdefinition

To determine the equivalence of two running tasks, we therefore need to look at all possible user interactions, and check that they affect the two tasks in the same way.
To do this, we need a way to talk about sequences of input events, instead of just single input events.
We give the following definition for this:

\startdefinition[def:input-seq]{Input sequences}
  An input sequence $I = \iota_1 \cdot \ldots \cdot \iota_j$ is a finite sequence of input events.
  Given a task $t_0 : \Task\ \tau$ and a state $\sigma_0$, we say that $I$ is a \emph{valid} input sequence for $t_0$ in $\sigma_0$ if and only if:
  \startformula
    t_0,\sigma_0 \interact{\iota_1} t_1,\sigma_1 \interact{\iota_2} \ldots \interact{\iota_j} t_{j},\sigma_{j}
  \stopformula
  with $\iota_i \in \Inputs(t_{i-1})$, for all $\iota \in \{1, \ldots, j\}$.

  We will use the shorthand notation $t_0,\sigma_0 \execute{I} t_{j},\sigma_{j}$ to denote the above derivation.
  We also consider the empty input sequence, denoted by $\epsilon$, to be a valid input sequence,
  and for all tasks $t$ and states $\sigma$ we have that: $t,\sigma \execute{\epsilon} t,\sigma$.
\stopdefinition

We will make a distinction between running tasks that forever remain running, no matter what inputs you send to it;
and running tasks for which there exists at least one input sequence which escapes, i.e. which transitions to another task condition.
We will call the former class \emph{looping} tasks, and the latter class \emph{branching} tasks, formally:

\startdefinition{Looping and branching}
  We call a running task $t : \Task\ \tau$ in state $\sigma$ \emph{looping} if and only if for all valid input sequences $I$
  we have that $t,\sigma \execute{I} t',\sigma'$ and $t'$ is again a running task.
  If there exists at least one valid input sequence for which $t'$ is not a running task, then we call $t$ \emph{branching}.
\stopdefinition

For branching tasks, it is possible to transition to either a finished or a stuck task condition.
Take for example the task $\Enter^k \Int \Step \lambda x.\ t$, which is a running task that transitions to $t$ after a valid input event.
So long as $t$ is not failing, the step can be taken, and so $t$ can be any non-failing task.
It is also possible that for some input events, it transitions to one task condition, and for others, that it transitions to a different task condition.
For example the task $\Enter^k \Int \Step \lambda x.\ \Ite{x \leq 2}{t}{t'}$ transitions to $t$ for some inputs, and to $t'$ for other inputs.
However, a running task can never transition to a failing task.

\startproposition[prp:running-not-to-failing]{Running tasks will not fail}
  If $t : \Task\ \tau$ is a running task,
  then for all states $\sigma$ and for all valid input sequences $I$,
  we have that
  \NL if $t,\sigma \execute{I} t',\sigma'$
  \NL then $\neg \Failing(t')$.
\stopproposition

\startproof
  By assumption $t$ is running, and therefore itself not failing,
  and by the definition of the normalisation rules, $t$ cannot make a transition to a failing task.
\stopproof

We say that two running tasks $t_1$ and $t_2$ in state $\sigma$ are semantically equivalent if for all valid input sequences $I$
we have that $t_1,\sigma \interact{I} t_1',\sigma' \iff t_2,\sigma \interact{I} t_2',\sigma'$, with $\Value(t_1',\sigma') = \Value(t_2',\sigma')$, and $\Inputs(t_1') = \Inputs(t_2')$.
That is, for all possible user interactions, $t_1$ and $t_2$ are not observably different.
Because of \see[prp:running-not-to-failing], we do not need to check whether the tasks fail or not.


\stopcomponent
