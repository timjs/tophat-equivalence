\startcomponent interplay
\product thesis
\environment thesislayout


\section{Interplay}

In this section we prove some properties of our observations functions introduced in \see[sec:observations]
and show that they play well together.


\subsection{Failing}

First, we prove that our failing observation $\Failing$ indeed observes a task that
does \emph{not} have a value, and does \emph{not} have possible inputs.
That is, the value observation $\Value$ is undefined,
and the inputs observation $\Inputs$ gives the empty set.

\startproposition[prp:failing-no-interaction]{Failing tasks cannot have interaction}
  For all well typed normalised tasks $n$ and state $\sigma$,
  % that is $\Gamma,\Sigma \infers n:\Task\tau$ and $\Gamma,\Sigma \infers \sigma$,
  we have that
  \NL if $\Failing(n) = \True$,
  \NL then $\Value(n,\sigma) = \bot$ and $\Inputs(n) = \emptyset$.
\stopproposition

\startproof
  This proposition is mechanically proved by induction over $n$ in Idris,
  see\-\impl{Task.Proofs.Failing.failingMeansNoInteraction}.
\stopproof

Note that the converse statement is not true:
a task which does not have a value and does not have possible inputs is not necessarily failing.
The task \TS{view 42 <&> fail}
does not have a value, because the right hand side of the pair combinator does not have a value; and
does not have possible inputs, because the left hand side is a read-only editor.
However, $\Failing(\View 42 \Pair \Fail) = \False$,
because the failing observation needs both sides of the pair to be failing, but the left hand side clearly is not.
This deviates from the properties from earlier \TOPHAT\ versions \cite[conf/ppdp/SteenvoordenNK19,conf/ifl/NausSK19].


\subsection{Inputs}

Next, we validate our inputs observation $\Inputs$.
It calculates all possible inputs for a given task.
We need to show that the set of possible inputs it produces is both sound and complete with respect to the handling ($\handle{}$) and interaction ($\interact{}$) semantics.
By sound we mean that all inputs in the set of possible inputs can actually be handled by the semantics,
and by complete we mean that the set of possible inputs contains all inputs that can be handled by the semantics.
\see[prp:correctness-i] expresses exactly this property for the handling semantics.

\startproposition[prp:correctness-i]{Inputs is sound and complete (wrt handle)}
  % For all well typed expressions $e$, states $\sigma$, and inputs $i$,
  For all well typed normalised tasks $n$, states $\sigma$, and inputs $\iota$,
  % that is $\Gamma,\Sigma \infers n:\Task\tau$ and $\Gamma,\Sigma \infers \sigma$,
  we have that
  \NL $i \in \Inputs(n)$ if and only if
  \NL there exists task $t'$, state $\sigma'$, and dirty set $\delta'$
  \NL such that $n,\sigma \handle{\iota} t',\sigma',\delta'$.
\stopproposition

\startproof
  This theorem is mechanically proved by induction on $n$ in Idris,
  see\-\impl{Task.Proofs.Inputs.inputIsHandled} and \impl{Task.Proofs.Inputs.handleIsPossible}.
\stopproof

Because of the structure of the interaction semantics ($\interact{}$),
\see[prp:correctness-i] directly gives us soundness and completeness of inputs with respect to interact.

\startcorollary[cor:correctness-i]{Inputs is sound and complete (wrt normalise)}
  For all well typed normalised tasks $n$, states $\sigma$, and inputs $\iota$,
  % that is $\Gamma,\Sigma \infers n:\Task\tau$ and $\Gamma,\Sigma \infers \sigma$,
  we have that
  \NL $i \in \Inputs(n)$ if and only if
  \NL there exists normalised task $n'$, and state $\sigma'$
  \NL such that $n,\sigma \interact{\iota} n',\sigma'$.
\stopcorollary

By combining \see[prp:failing-no-interaction] and \see[cor:correctness-i],
we know that failing tasks cannot handle input.

\startcorollary[cor:failing-tasks-no-handling]{Failing tasks cannot handle input}
  For all well typed normalised tasks $n$ and states $\sigma$,
  % that is $\Gamma,\Sigma \infers n:\Task\tau$ and $\Gamma,\Sigma \infers \sigma$,
  we have that
  \NL if $\Failing(n)$,
  \NL there is no input $\iota$ such that $n,\sigma \handle{\iota} t',\sigma',\delta'$.
\stopcorollary


\subsection{Value}

Last, we prove that, when we can observe a value from a task, this task is a \emph{static task}.
By static tasks, we mean that the shape of the tree of task nodes and leaves are static and cannot be dynamically altered at runtime.
\doifthesis{We will use this property when discussing equivalence of task in \see[chp:equivalence].}
% \See[fig:static-tasks-grammar] shows its grammar.

\placefigure[][fig:static-tasks-grammar]
  {Static tasks}
  {\startgrammar
    \useproduction{G-Static-Tasks}
  \stopgrammar}

\startdefinition[def:static-task]{Static task}
  We call tasks \emph{static} when they consist of
  a valued editor,
  pairs of static tasks,
  or transforms of static tasks.
  That is, they confirm to the grammar presented in \see[fig:static-tasks-grammar].

  Note that static tasks are a subset of normalised tasks.
\stopdefinition

\startproposition[prp:valued-means-static]{Valued tasks are static}
  For all well typed normalised tasks $n$ and states $\sigma$,
  % that is $\Gamma,\Sigma \infers n:\Task\tau$ and $\Gamma,\Sigma \infers \sigma$,
  we have that
  \NL if $\Value(n,\sigma) = v$,
  \NL then $n$ is static.
  % Moreover, we have that
  % \NL the shape of $n$ cannot be altered at runtime.
\stopproposition

\startproof
  We have that $\Value(n,\sigma) = v : \beta$.
  When taking the definition of $\Value$ into account (as given in \see[fig:observation-value]),
  we see that $n$ can only be in one of four shapes:
  \startitemize
    \item a valued editor $d$ (that is $d \neq \Enter\beta$);
    \item a pair $n_1 \Pair n_2$, for which $\Value(n_1,\sigma) = v_1$ and $\Value(n_2,\sigma) = v_2$;
    \item a transform $e_1 \Trans n_2$, for which $\Value(n_2,\sigma) = v_2$;
    \item a lift $\Lift v$.
  \stopitemize
  Especially, step ($n_1 \Step e_2$) and fail ($\Fail$) are ruled out, because they have no value.
  Choice ($n_1 \Choose n_2$) is ruled out, because by rules \seerule{N-Choose[Left,Right,None]},
  the only way that this combinator can survive normalisation is for both $n_1$ and $n_2$ to not have a value,
  which should be the case here.
  Now, by induction, we have that $n$ is static and meets the grammar in \see[fig:static-tasks-grammar].

  See also \impl{Task.Proofs.Static.valued\_means\_static} in Idris.
\stopproof

As static tasks do not contain steps ($n_1 \Step e_2$) or choices ($n_1 \Choose n_2$),
their form cannot be altered at runtime.
That is, there is no way for end users to create new (sub)tasks.

\startcorollary[cor:static-stays-static]{Static tasks stay static}
  For all well typed normalised tasks $n$ and states $\sigma$,
  % that is $\Gamma,\Sigma \infers n:\Task\tau$ and $\Gamma,\Sigma \infers \sigma$,
  we have that
  \NL if $n$ is static,
  \NL its shape cannot be altered at runtime.
\stopcorollary

% \startproof
%   By induction on the handling and normalisation rules.
%   \todo{Idris}
% \stopproof

% Now, because we cannot create new (sub)tasks at runtime,
In particular, we cannot create or delete editors at runtime.
So the input events a static task can process will always be the same.

\startcorollary[cor:static-keeps-input]{Static tasks keep input observation}
  For all well typed normalised tasks $n$ and states $\sigma$,
  % that is $\Gamma,\Sigma \infers n:\Task\tau$ and $\Gamma,\Sigma \infers \sigma$,
  we have that
  \NL if $n$ is static,
  \NL and $n,\sigma \interact{\iota} n',\sigma'$ for some $\iota \in \Inputs(n)$
  \NL then $\Inputs(t) = \Inputs(t')$.
\stopcorollary

\stopcomponent