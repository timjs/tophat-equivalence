\startcomponent expressions
\product thesis
\environment thesislayout

\section[sec:exprequiv]{Expression equivalence}

Before we consider semantic equivalence of tasks, we first look at semantic equivalence of expressions in the host language.
Let us start with an example.

\startexample[exm:expression-equivalence]{Expression equivalence}
  Consider the following two expressions:
  \startformulas[align]
    \NC e_1 \NC:=\NC \lambda x:\Int.\ \Ite{x < 0}{-x}{x}    \NR
    \NC e_2 \NC:=\NC \lambda y:\Int.\ \Ite{y \geq 0}{y}{-y} \NR
  \stopformulas
  \noindentation
  It should be obvious that these two functions are equivalent: they both return the absolute value of their argument.
  Therefore, we should be able to use them interchangeably within any \TOPHAT-program without changing its behavior.
  So even though the functions $e_1$ and $e_2$ are different, we will never detect a difference between them when they are being used within a \TOPHAT-program, because for all possible arguments, $e_1$ and $e_2$ evaluate to the same result.
\stopexample

When deciding if two expressions in the host language are equivalent, it is not enough to just look at the resulting value after evaluation.
We need to consider all \emph{contexts} that an expression can be used in.
This leads to the definition of \emph{contextual equivalence}.
\inlinecite[conf/ac/Pitts00] defines contextual equivalence informally as follows:

\startquotation
  Two phrases of a programming language are \emph{contextually equivalent} if
  any occurrences of the first phrase in a complete program can be replaced
  by the second phrase without affecting the observable results of executing
  the program.
\stopquotation

\doifthesis{
  \placefigure[bottom][fig:contexts-grammar]
    {Context grammar for expressions}
    {\startgrammar
      \useproduction{G-Expression-contexts}
    \stopgrammar}
}

This kind of equivalence is also called \emph{operational}, or \emph{observational} equivalence.
To formally define such a notion of contextual equivalence for a given programming language, we must answer two questions:
\startitemize[n,compact]
  \item What is a \emph{complete program}?
  \item What are the \emph{observable results}?
\stopitemize
Depending on the answers to these two questions,
this can result in different definitions of semantic equivalence for the same programming language \cite[conf/ac/Pitts00].
%
For expressions in \TOPHAT, we answer these two questions as follows:
\startitemize[n]
  \item We will consider an expression in the host-language a complete program if it does not contain any free variables.
  \item The only observation we are interested in is the resulting value after evaluation.
\stopitemize

\doifthesis{
  \placefigure[top][fig:contexts-tasks-grammar]
    {Context grammar for tasks}
    {\startgrammar
      \useproduction{G-Pre-task-contexts}
      \useproduction{G-Pre-editor-contexts}
    \stopgrammar}
}

We need a way to substitute an expression in a program by another.
For this, we use the notion of an expression context.
An \emph{expression context} $E[\cdot]$ is a complete program that can contain holes, denoted by the symbol $\cdot$, which can be filled.
We write $E[e]$ for the expression that results from replacing all occurrences of $\cdot$ in $E$ by $e$.
So, we will replace all holes with the same expression $e$.
In the most common case, there will be just one hole in $E$ to be filled.
\doifthesiselse{
  \seein{Figures}[fig:contexts-grammar] \seein{and}[fig:contexts-tasks-grammar] give the grammars for expression-, pre-task-, and pre-editor-contexts.
}{
  \see[fig:contexts] gives the grammars for expression- and task-contexts.
}
% Note that the grammar for expression contexts does not allow contexts under lambdas.
% Otherwise we would be able to construct open expressions with free variables,
% violating point\~1.

\doifpaper{
  \placefigure[][fig:contexts]
    {Context grammar}
    {\sidebyside
      {\usemacro{G-Expression-contexts}}
      {\usemacro{G-Task-contexts}}
    }
}

\doifthesis{
  The resulting values of \TOPHAT\ programs are those given in the value grammar given in \see[fig:value-host-grammar].
}
If we would allow contexts $E$ to be of every available type,
we would have the same problem as introduced in \see[exm:expression-equivalence].
Then, $E$ can also be a lambda function and we can only determine that two lambda functions are equivalent by considering all contexts that they can be used in.
To avoid a circular definition,
our definition of expression equivalence will only quantify over contexts of basic types $\beta$.
Indeed, we can only directly observe the equivalence of two basic values.

Taking this together, this leads to the following definition of expression equivalence:
\startdefinition[def:expressions-eq]{Expression equivalence ($\exprequiv$)}
  Given two expressions $e_1, e_2 : \tau$
  where $\tau \neq \Task\ \tau'$ for some $\tau'$,
  we say that $e_1$ and $e_2$ are semantically equivalent,
  written $e_1 \exprequiv e_2$
  \NL if for all contexts $E[\cdot] : \beta$ where $\cdot : \tau$, and for all values $b : \beta$
  \NL we have that $E[e_1] \evaluate b$ if and only if $E[e_2] \evaluate b$
\stopdefinition

Actually proving that two expressions are contextually equivalent is hard, as we would need to quantify over all contexts.
That is, we would need to consider all possible ways that a program can use an expression.
However, showing that two expressions $e_1$ and $e_2$ are contextually \emph{in}equivalent is straightforward.
All we have to do is find one context $E[\cdot]:\beta$ such that $E[e_1] \evaluate b_1$ and $E[e_2] \evaluate b_2$ with $b_1 \neq b_2$.

\startexample[exm:expr-inequiv]{Expression inequivalence}
  The expressions
  \startformulas[align]
    \NC e_1 \NC:=\NC \lambda x:\Int.\ \Ite{x < 0}{2}{3} \NR
    \NC e_2 \NC:=\NC \lambda x:\Int.\ \Ite{x > 0}{3}{2} \NR
  \stopformulas
  \noindentation
  are not contextually equivalent.
  Take the context $E[\cdot]:\Int$ with $E[\cdot] =\ \cdot\ 0$, then:
  \startformulas[align]
    \NC E[e_1] \NC=\NC (\lambda x:\Int.\ \Ite{x < 0}{2}{3})\ 0 \evaluate 3 \NR
    \NC E[e_2] \NC=\NC (\lambda x:\Int.\ \Ite{x > 0}{3}{2})\ 0 \evaluate 2 \NR
  \stopformulas
  \noindentation
  So we found a context for which both expressions evaluate to a different (basic) value.
\stopexample


\stopcomponent