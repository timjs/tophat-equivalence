\section[sec:tophat]{The TopHat language}

\TOPHAT\ is a formal language for reasoning about task-oriented programs.
It is described by a layered operational semantics, consisting of multiple big-step semantic functions for reducing expressions, and two labelled transition systems for handling user inputs.
Its main semantics are \emph{evaluation}, \emph{fixation}, and \emph{interaction}.
To make clear which features come from \TOP\ and which features come from functional programming, \TOPHAT\ is separated into a task language and an underlying host language \cite[conf/ppdp/SteenvoordenNK19,Steenvoorden22].

The host language of \TOPHAT\ is a simply typed $\lambda$-calculus,
extended with with basic types $\beta$, references $h$, and the task type constructor $\Task$.
Basic types are the unit type; primitive types for booleans, integers, and strings; and pairs of other basic types.
% Additionally, there are basic types $\beta$, which contain only a subset of all types $\tau$.
%
We will present the main components of the task language, which is embedded into the host language, in the next subsections.
% Finally, \see[sec:tophat-examples] will discuss some larger examples.


\subsection{Editors}

An editor is a task holding a value.
Editors allow end users to interact with the system by entering and changing information.
When a user sends an input event to an editor, the editor will update its current value to reflect the change.
There are no output events.
Instead, the current value of an editor can be observed and used in subsequent tasks.
All editors and input events are \emph{keyed} by a unique key $k$.
This is to identify which editor needs to handle which user input.
There are three types of editors in \TOPHAT:

\startitemize
  \item
    \emph{Empty editors} or \emph{unvalued editors} ($\Enter^k \beta$) are editors that currently hold no value.
    They can be seen as an input prompt to the user to enter data.
    Empty editors are annotated with a basic type $\beta$, which means that only basic values of type $\beta$ are accepted by the editor.
    Once an empty editor receives a valid input event, it becomes a filled editor containing the new data.

  \item
    \emph{Filled editors} or \emph{valued editors} ($\Update^k b$) are editors that currently hold the basic value~$b$.
    Filled editors can be seen as either outputting a value, or as an input prompt that comes with a default value.
    They can never be cleared, only updated with new values of the same type.

  \item
    \emph{Shared editors} ($\Change^k h$) watch heap locations $h$.
    They allow the user to view and change shared values.
    Whenever a shared editor is updated, all shared editors watching the same heap location $h$ will be updated as well.
    To create a heap location to an expression $e$, we write $\Share e$.
    Reference assignment ($e_1 \Assign e_2$) allows the system to assign a new value to a heap location.
\stopitemize

% \placefigure[b][fig:editor-state]
  {Possible states of editors and its transitions}
  {\startpicture[x=1pc,y=1pc,line width=1pt,>={Triangle},align=center]
    \node (unvalued) [circle,draw,minimum size=3pc,very thick]                     {\emph{un-}\\\emph{valued}};
    \node (valued)   [circle,draw,minimum size=3pc,very thick,right=4 of unvalued] {\emph{valued}};
    \node (shared)   [circle,draw,minimum size=3pc,very thick,right=6 of valued]   {\emph{shared}};
    \path (unvalued) edge[->] node[above] {Enter} (valued);
    \path (valued.north) edge[->,out=60,in=30,looseness=6] node[above right] {Update} (valued.north east);
    \path (shared.north) edge[->,out=60,in=30,looseness=6] node[above right] {Change} (shared.north east);
  \stoppicture}


Filled editors and shared editors both have read-only variants ($\View^k b$, $\Watch^ h$).
End users can see the value in a read-only editor, but not change it.
As editors can only handle basic values of type $\beta$,
there is also a lift combinator ($\Lift v$) which lifts any value $v$ into the task world.

% The relation between the different editors is illustrated in the state diagram in Figure \ref{fig:editor-states}.
% Since editors are already fully reduced, no normalisation needs to be done.
% Figure \ref{fig:n-editors} gives the normalisation rules for editors.


% \subsection{Combinators}
\subsection{Sequential composition}

The step combinator ($\Step$) allows the result from one task to determine the next task.
We call this \textit{sequential composition}.
The step combinator expects a task $t$ of type $\Task\ \tau_1$ on the left hand side,
and a continuation $e$ on the right hand side, which is a function from from $\tau_1$ to a successor task of type $\Task\ \tau_2$.
Steps are guarded.
A step can only be taken if two conditions are met: (1) the task on the left-hand side has an observable value;
and (2) the evaluation of the continuation on the right-hand side with this value does not fail.
A failing task ($\Fail$) stands for an impossible task.
A task that is failing never has a value and never accepts input.
% The normalisation rules for step are given in Figure \ref{fig:n-step}.
% There are three cases: either the first condition fails, and the step remains guarded (\textsc{N-StepNone}); or the first condition is met, but the second condition fails, and the step remains guarded (\textsc{N-StepFail}); or both conditions are met, and the step can proceed (\textsc{N-StepCont}).
% Note that the last two rules use evaluation in the host language to compute the successor task.
% The result of this evaluation is only used when the step can be taken successfully (\textsc{N-StepCont}), and discarded otherwise (\textsc{N-StepFail}). \\

\startexample[exm:coffee-machine]{Coffee machine}
  Consider the following task-program:
  \begin{TASK}
    let coffeeMachine : Task Drink = enter Int >>= \x.
      if x == 1 then lift Coffee else if x == 2 then lift Tea else fail
  \end{TASK}
  \noindent
  This program describes a coffee machine that can either serve coffee or tea.
  Coffee is served when one coin is inserted, and tea is served when two coins are inserted.
  For any other number of coins, the step remains guarded by the failing task ($\Fail$).
  This means that the coffee machine returns nothing and waits until a correct number of coins is inserted.
\stopexample

The transform combinator ($\Trans$) maps a function over a task.
It takes a function of type $\tau_1 \to \tau_2$ on the left-hand side,
and a task of type $\Task\ \tau_1$ on the right-hand side, resulting in a task of type $\Task\ \tau_2$.
% The normalisation rule of the transform combinator, given in Figure \ref{fig:n-trans}, does not actually do anything.
If we want to apply the function and use the result, we would need to use the transform combinator in combination with the step combinator,
as we can see in the example below.
% This would allow us to extract the value of the transform task by using the task observation $\Value$, which in case of the transform combinator, is defined as $\Value(e \Trans t) = v'$ when $\Value(t,\sigma) = v$ and $ev \eval v'$.
% So, if the task on the right-hand side has a value $v$, and applying the function $e$ to $v$ results in the value $v'$, then the transform combinator has the value $v'$.

\startexample{Traffic light}
  Let us consider a simple example:
  \begin{TASK}
    let trafficLight : Task Light =
      ((\x. if x then Green else Red) <*> enter Bool) >>= \y. view y
  \end{TASK}
  This program describes a traffic light whose light is initially turned off, but given the right input, it can either become red or green.
  So long as no input is given, the transform task on the left-hand side of the step combinator has no value, and the step remains guarded.
  Once an input is entered, transform returns the value \ctr{Green} if $\True$ was entered, and \ctr{Red} if $\False$ was entered, upon which the step proceeds and displays the result.
  % Now that the left-hand side of step has a value, it proceeds and displays the value.
  % upon which the step can proceed.
  % Once either $\True$ or $\False$ is entered, the transform task gets the value $\Green$ or $\Red$
\stopexample


\subsection{Parallel composition}

Pairing ($\Pair$) combines the result of two tasks, but only if both branches have a value.
If the left task is of type $\Task\ \tau_1$, and the right task is of type $\Task\ \tau_2$, then their pairing is of type $\Task\ (\tau_1 \times \tau_2)$.
However, if one or both branches have no value, then the resulting task also has no value.

Choosing ($\Choose$) chooses one of two branches.
Therefore, both branches should be of the same type $\Task\ \tau$.
This combinator is left-biased: it returns the leftmost task that has a value.
If neither task has a value, then the resulting task also has no value.
% The normalisation rules for pairing and choice are given in Figure \ref{fig:n-parallel}.
See also \see[fig:observation-value] for the definition of the value observation for pairing and choice.

These combinators allow user to work on two tasks in parallel, but unlike the name suggests, parallel does not mean that there is non-determinism.
The order of execution is determined by the order of user inputs send.
Instead, parallel here means that the order in which we execute the tasks, and their subtasks, does not matter.

\startexample{Breakfast}
  Let us consider the following task, which makes use of both parallel combinators:
  \begin{TASK}
    let make : $\tau$ -> Task $\tau$ = \x. enter Unit >>= \y. view x in
    let makeBreakfast : Task (Drink * Food) =
      ((make Tea <|> make Coffee) <&> make Egg) >>= eatBreakfast
  \end{TASK}
  This program describes a simple workflow for making breakfast.
  Breakfast consists of something to drink (tea or coffee), and something to eat (eggs).
  The drink and the food are prepared in parallel ($\Pair$), which means that the order in which they are made does not matter.
  For the drink, users have a choice ($\Choose$) whether they want tea or coffee with their breakfast.
  For the food, users will always make an egg.
  We will use the function \var{make} to simulate that the user must first perform an action (i.e. send user input) before an item is prepared and the task has a value.
  Only when both the drink and food are ready, can the step be taken and can we enjoy our breakfast.
\stopexample


\subsection[sec:observations]{Observations}

Not just $\Fail$ can fail, tasks with failing subtasks can also fail.
For example, the pairing $\Fail \Pair \Fail$ is also a failing task.
It is equivalent to $\Fail$.
To capture what tasks are failing we introduce the failing observation $\Failing$, which is defined in \see[fig:observation-failing-inputs].
Failing is especially useful when used in combination with the step combinator,
as we have seen in \see[exm:coffee-machine].

% \placefigure[][fig:o-failing]
%   {Failing observation on tasks}
%   {\usemacro{O-Failing}}

\placefigure[][fig:observation-failing-inputs]
  {Failing and inputs observations on tasks}
  {\sidebyside[0.45][0.55]
    {\usemacro{O-Failing}\vspace*{\fill}}
    {\usemacro{O-Inputs}}}


% \subsection{Input}

Once a \TOPHAT\ program is normalised, it is possible to send input to it.
The input event $k!b$ indicates that the input $b$ should be entered into the editor with key $k$.
To do this, it is useful to know what inputs a given task accepts.
The observation function $\Inputs$ returns the set of input events that are currently possible for a given task.
We consider an input event $\iota$ a \emph{valid} input event for the task $t$ iff $\iota \in \Inputs(t)$.
Its definition is given in \see[fig:observation-failing-inputs].
For the three read-write editors, $\Inputs$ returns all input events $k!b$ where $b$ is of the correct type.
For task combinators, $\Inputs$ is defined recursively.
For all other tasks, $\Inputs$ returns the empty set.
$\Inputs$ is only defined on \emph{normalised tasks},
as those tasks are ready to receive user input.
We'll discuss this in more detail in the next section.

% \placefigure[][fig:o-inputs]
%   {Inputs observation on tasks}
%   {\usemacro{O-Inputs}}

To be able to normalise tasks, we need to have a way to determine the value of tasks.
For this, we introduce the task observation $\Value$.
Given a task $t : \Task\ \tau$ and its current state $\sigma$, this function returns the task's value $v$ of type $\tau$.
It is also possible that a task's value is undefined, in which case we write $\Value(t,\sigma) = \bot$.
The definition of $\Value$ is given in \see[fig:observation-value].

\placefigure[][fig:observation-value]
  {Value observation on tasks}
  % {\usemacro{O-Value}}
  {\sidebyside[0.3][0.7]
    {\usemacro{O-Value-A}}
    {\usemacro{O-Value-B}}}


\subsection{Semantics}

\see[fig:semantic-functions] shows all semantic relations as used by \TOPHAT s formalisation of tasks
together with the task observations.
They fit into three layers: the \emph{host layer}, the \emph{internal layer}, and the \emph{external layer}.
Next, we will discuss all semantic arrows starting from the bottom layer.

% \section{Overview of semantic relations and functions}

\placefigure[][fig:semantic-functions]
  {Semantic functions and their relation
  (read $x \bullet\!\!\lowerbox{0.1ex}{\mbox{–}}\ y$ as `$x$ makes use of $y$')}
  {\startpicture[x=1pc,y=0.8pc,line width=1pt,*/.tip=Circle,align=center]
    \node (h)  at ( 3,  0) {handle ($\handle{}$)};
    \node (oi) at (10,-1.5){inputs ($\Inputs$)};
    \node (i)  at ( 3, -3) {interact ($\interact{}$)};
    \draw[dashed,thin] (-9,-4.5) -- (14,-4.5);
    \node (f)  at ( 3, -6) {fixate ($\fixate$)};
    % \node (ow) at (10, -6) {watching ($\Watching$)};
    \node (n)  at ( 3, -9) {normalise ($\normalise$)};
    \node (ov) at (10, -9) {failing ($\Failing$)\\ value ($\Value$)};
    \draw[dashed,thin] (-9,-10.5) -- (14,-10.5);
    \node (e)  at ( 3,-12) {evaluate ($\evaluate$)};
    %
    % \node     at (3,-14) {(\small read $x \bullet\!\!\lowerbox{0.1ex}{\mbox{–}}\ y$ as `$x$ makes use of $y$')};
    %
    \node[anchor=west] (el) at (-8,-1.5) {\emph{external layer}};
    \node[anchor=west] (il) at (-8,-7.5) {\emph{internal layer}};
    \node[anchor=west] (hl) at (-8,-12)  {\emph{host layer}};
    %
    \draw[-*] (e) -- (n);
    \draw[-*] (n) -- (f);
    \draw[-*] (f) -- (i);
    \draw[-*] (h) -- (i);
    % \draw[-*] (ow) -- (f);
    \draw[-*] (ov) -- (n);
  \stoppicture}

Evaluating terms in the host language to values is handled by the big-step \emph{evaluation} semantics,
where a value is an expression in the host language that cannot be reduced further.
Values $v$ can be lambda functions, pairs of values, unit, constants, locations, or tasks.
Basic values $b$ are a subset of values $v$ that are of basic type $\beta$.
We denote the evaluation of expression $e$ to the value $v$ by $e \evaluate v$.

After evaluation is done, a task is ready to be \emph{normalised}.
Normalisation is a big-step semantics that is responsible for reducing tasks until they are ready to accept input.
We write $t, \sigma \normalise t', \sigma', \delta'$ to denote the normalisation of task $t$ in state $\sigma$ to task t' in state $\sigma'$.
The state $\sigma$ is a mapping from heap locations to basic values.
It keeps track of all references created so far, and what value they currently hold.
Additionally, normalisation returns a set $\delta'$, which we will call a dirty set,
containing all heap locations whose value has been changed while normalisation took place.
It will be used in the \emph{fixation} semantics, which is defined on top of normalisation.
Due to mutable references, the fixation semantics is required to make sure that a task is fully normalised before any user interaction is allowed.
Fixation makes use of normalisation as a helper semantics.
We write $t, \sigma, \delta \fixate t', \sigma'$ to denote the fixation of task $t$ in state $\sigma$ given the dirty set $\delta$, to task $t'$ in state $\sigma'$.

Input events are handled by the handling and interaction semantics.
Both semantics are small-step semantics that take an user input $\iota$ for each step.
For the \emph{handling} semantics, we write $t,\sigma \handle{\iota} t',\sigma',\delta'$
to denote that handling the user input $\iota$ in task $t$ and state $\sigma$ results in task $t'$ and state $\sigma'$.
Similar to the normalisation semantics, the handling semantics also returns a dirty set $\delta'$.
The \emph{interaction} semantics then makes use the helper handling semantics,
and the previously mentioned fixation semantics to make sure that, after user interaction,
a task is fully reduced and ready to accept the next input.
For the interaction semantics, we write $t,\sigma \interact{\iota} t',\sigma'$ to denote that task $t$ in state $\sigma$ transitions to task $t'$ in state $\sigma'$ after the user input $\iota$.
For further details we refer to \cite[Steenvoorden22].

% %, whose rules are given in the appendix.
% Most handling rules simply pass along the input events to their subtasks.
% The only interesting rules are the handling rules for editors.
% A valid input event to an empty editor results in a filled editor containing the new data (\seerule{H-Enter}),
% a valid input event to a filled editor updates its value (\seerule{H-Update}),
% and a valid input event to a shared editor updates the state $\sigma$ such that the heap location $h$ now contains the new value (\seerule{H-Change}).
% It also returns $\delta = \set{h}$, which will later be used in the fixation semantics described in Section~\ref{sec:references}.
